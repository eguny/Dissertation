
{\actuality} Изучение явлений порядок-беспорядок --- фундаментальная задача равновесной статистической механики. Были приложены большие усилия, чтобы понять основные механизмы, ответственные за спонтанное упорядочение, а также природу фазовых переходов во многих типах систем. В частности, в течение последних лет большое внимание уделялось фрустрированным моделям~\cite{liebmann1986}. Термин «фрустрация» был введен~\cite{toulouse1977,vannimenus1977} для описания ситуации, когда спин (или количество спинов) в системе не может найти конфигурацию, полностью удовлетворяющей всем взаимодействиям с соседними спинами. Этот определение может быть применено к модели Изинга, моделям Поттса и векторным спинам. Как правило, фрустрация вызвана либо конкурирующими взаимодействиями или же решеточной структурой, как в треугольной, гранецентрированной кубической (ГЦК) и гексагонально-плотноупакованной (ГПУ) решетках с антиферромагнитным взаимодействием ближайших соседей. Фрустрационные эффекты чрезвычайно богаты, многие из которых в настоящее время еще не изучены.

Помимо того, что настоящие магнитные материалы часто не подходят из-за нескольких видов взаимодействий, фрустрированные спиновые системы имеют свой собственный интерес в статистической механике. Недавние исследования показывают, что многие известные статистические методы и теории столкнулись со многими трудностями при работе с фрустрированными системами. В некотором смысле фрустрированные системы - отличные кандидаты для тестирования приближений и улучшения теории. Поскольку механизмы многих явлений не поняты в реальных системах (неупорядоченные системы, системы с дальним взаимодействием, трехмерными системами и т.д.) стоит искать истоки этих явлений в точно решаемых системах. Эти точные результаты помогут качественно понять поведение реальных систем, которые в целом намного сложнее.
Однако, далеко не каждая физическая задача может быть решена точно. Такая ситуация скорее является исключением из правил, поскольку точное решение задачи связано с большим количеством сложных математических препятствий, которые чаще всего преодолеть не представляется возможным. Зачастую задачи такого типа решаются применением тех или иных приближений, которые значительно упрощают рассматриваемую задачу. Однако за этим упрощением стоит потеря значимой для исследователя информации.

Модель Изинга является простейшей задачей теории ферромагнетизма, решение которой в одномерном и двумерном случае можно получить точно~\cite{baxter1985}. Решение одномерной модели Изинга получил сам Изинг еще в 1925 году~\cite{ising1925}. Решению модели на двумерных решетках послужило опубликование Крамерсом и Ваннье статьи~\cite{kramers_wannier1, kramers_wannier2}, в которой вводится так называемая трансфер-матрица. С помощью трансфер-матрицы авторы сначала переполучили результат Изинга на одномерной цепочке, после чего получили точное выражение для температуры фазового перехода на квадратной решетке. Впоследствии, этим результатом воспользовался Онзагер в 1941 году для получения точного решения модели Изинга на квадратной решетке~\cite{onsager1941}, которое, в свою очередь, привело к получению решений на других решетках (треугольная решетка --- Ваннье~\cite{wannier1950}, гексагональная решетка --- Гутаппель~\cite{houtapell1950}, кагоме --- Кано и Найя~\cite{kano_naya1953}), а также бурному развитию физики критических явлений~\cite{yang1952, brush1967, mussardo2010}. Однако, решение Онзагера оказалось чрезвычайно сложным, поэтому исследователи вскоре изобрели несколько иных подходов к решению модели Изинга на двумерной решетке~\cite{kaufrman1949, hurst1960, kasteleyn1963}. Одним из таких, является комбинаторный метод~\cite{kac1952, montroll1953, vdovichenko1965}, концепцию которого предложил Фейнман~\cite{feynmann1978}. 

Так или иначе, в обоих методах авторы рассматривали обычную (классическую) модель Изинга, то есть обменное взаимодействие между спинами в решетке вдоль одного направления остается одинаковым. 

В настоящей работе предлагается рассмотреть обобщенную модель Изинга на одномерной цепочке с произвольным количеством различных обменных взаимодействий между спинами с учетом декорирования и магнитного поля, а также на квадратной решетке с двумя различными обменными взаимодействиями решетки как в горизонтальном, так и вертикальном направлениях.

Стоит заметить, что точного решения двумерной модели Изинга в магнитном поле даже на квадратной решетке до сих пор не найдено, поэтому в данной работе ограничимся рассмотрением обобщенной модели Изинга на квадратной решетке только в отсутствие магнитного поля.

Необходимо также отметить, что модель Изинга, как и любая другая математическая модель не привязана к какому-то конкретному материалу, соединению или эксперименту. К основным задачам таких моделей относятся выявление общих закономерностей в рассматриваемых процессах и явлениях, использование их в качестве прототипных для реальных трехмерных объектов и отклонение ложных догадок, сделанных на основе приближенных методов и экспериментальных результатов. 

{\aim} данной работы заключается в рассмотрении обобщенной	модели Изинга с произвольным количеством различных обменных взаимодействий как между ближайшими, так и между вторыми соседями с учетом декорирования и магнитного поля на одномерной цепочке, а также обобщенной модели Изинга с четырьмя различными обменными взаимодействиями между ближайшими соседями на квадратной решетке, с последующим изучением их термодинамических, магнитных и фрустрационных свойств.

Для достижения цели были поставлены следующие {\tasks}:
\begin{enumerate}[beginpenalty=10000] % https://tex.stackexchange.com/a/476052/104425
  \item Получить точное аналитическое решение обобщенной модели Изинга на одномерной цепочке при учете магнитного поля методом трансфер-матрицы Крамерса-Ваннье;
  \item Исследовать термодинамические, магнитные и фрустрационные свойства обобщенной модели Изинга на одномерной цепочке, в том числе при учете декорирования; 
  \item Получить точное аналитическое решение обобщенной модели Изинга на квадратной решетке комбинаторным методом Вдовиченко--Фейнмана;
  \item Исследовать термодинамические и фрустрационные свойства обобщенной модели Изинга на квадратной решетке.
\end{enumerate}

{\novelty}
\begin{enumerate}[beginpenalty=10000] % https://tex.stackexchange.com/a/476052/104425
  \item Впервые получено точное решение модели Изинга в магнитном поле на декорированной решетке.
  \item Комбинаторным методом Вдовиченко--Фейнмана были получены решения не только обобщенной модели Изинга на квадратной решетке, но и обычной модели Изинга на треугольной и гексагональной решетках.
  \item Исследованы термодинамические, магнитные и фрустрационные свойства обобщенной модели Изинга на одномерной цепочке и на декорированной цепочке.
  \item Изучены термодинамические и фрустрационные свойства обобщенной модели Изинга на квадратной решетке.
  \item Получены точные выражения для нуль-температурных энтропий и намагниченностей рассматриваемых моделей.
\end{enumerate}

{\influence} данной работы заключается в том, что полученные результаты могут быть использованы в качестве прототипных для изучения не только квазиодномерных, но и квазидвумерных фрустрированных спиновых систем, а также в возможности рассмотрения целого разнообразия так называемых декорированных решеток. Стоит подчеркнуть, что подавляющее большинство реальных решеток являются декорированными. Более того, некоторые кристаллические решетки можно назвать декорированными, а именно, решетки ГЦК и ОЦК, в сравнении с простой кубической решеткой (ОЦК декорирована по объему куба, а ГЦК декорирована по шести граням куба).

{\methods} \ldots

{\defpositions}
\begin{enumerate}[beginpenalty=10000] % https://tex.stackexchange.com/a/476052/104425
  \item Первое положение
  \item Второе положение
  \item Третье положение
  \item Четвертое положение
\end{enumerate}

{\reliability} полученных результатов обеспечивается строгой обоснованностью принятых приближений и допущений, использованием широко разработанных и обоснованных в мировой литературе аналитических и численных методов, а также тем фактом, что результаты находятся в согласии с теоретическими и экспериментальными работами других авторов.

{\probation}
Основные результаты работы докладывались на международных конференциях, семинарах и симпозиумах:
\begin{enumerate} [beginpenalty=10000]
	\item VII Euro-Asian Symposium «Trends in MAGnetism», 8-13 сентября 2019 года, Екатеринбург.
	\item XIII Международный семинар молодых ученых «Магнитные фазовые переходы»; 17 сентября 2019 года, Махачкала.
	\item Международная зимняя школа физиков-теоретиков «Коуровка–XXXVIII», 23-29 февраля 2020 года, «Гранатовая бухта», Верхняя Сысерть.
	\item 54-я Школа ПИЯФ по Физике Конденсированного Состояния (Школа ФКС-2020), 16-21 марта 2020 года, Гатчина:НИЦ «Курчатовский институт», Санкт-Петербург.
	\item VII Международная молодежная научная конференция Физика. Технологии. Инновации ФТИ-2020; 18-22 мая 2020 года, Екатеринбург.
	\item VIII Международная молодежная научная конференция Физика. Технологии. Инновации ФТИ-2021; 17-21 мая 2021 года, Екатеринбург.
\end{enumerate}

{\contribution} Автор принимал активное участие \ldots

\ifnumequal{\value{bibliosel}}{0}
{%%% Встроенная реализация с загрузкой файла через движок bibtex8. (При желании, внутри можно использовать обычные ссылки, наподобие `\cite{vakbib1,vakbib2}`).
    {\publications} Основные результаты по теме диссертации изложены
    в~XX~печатных изданиях,
    X из которых изданы в журналах, рекомендованных ВАК,
    X "--- в тезисах докладов.
}%
{%%% Реализация пакетом biblatex через движок biber
    \begin{refsection}[bl-author, bl-registered]
        % Это refsection=1.
        % Процитированные здесь работы:
        %  * подсчитываются, для автоматического составления фразы "Основные результаты ..."
        %  * попадают в авторскую библиографию, при usefootcite==0 и стиле `\insertbiblioauthor` или `\insertbiblioauthorgrouped`
        %  * нумеруются там в зависимости от порядка команд `\printbibliography` в этом разделе.
        %  * при использовании `\insertbiblioauthorgrouped`, порядок команд `\printbibliography` в нём должен быть тем же (см. biblio/biblatex.tex)
        %
        % Невидимый библиографический список для подсчёта количества публикаций:
        \phantom{\printbibliography[heading=nobibheading, section=1, env=countauthorvak,          keyword=biblioauthorvak]%
        \printbibliography[heading=nobibheading, section=1, env=countauthorwos,          keyword=biblioauthorwos]%
        \printbibliography[heading=nobibheading, section=1, env=countauthorscopus,       keyword=biblioauthorscopus]%
        \printbibliography[heading=nobibheading, section=1, env=countauthorconf,         keyword=biblioauthorconf]%
        \printbibliography[heading=nobibheading, section=1, env=countauthorother,        keyword=biblioauthorother]%
        \printbibliography[heading=nobibheading, section=1, env=countregistered,         keyword=biblioregistered]%
        \printbibliography[heading=nobibheading, section=1, env=countauthorpatent,       keyword=biblioauthorpatent]%
        \printbibliography[heading=nobibheading, section=1, env=countauthorprogram,      keyword=biblioauthorprogram]%
        \printbibliography[heading=nobibheading, section=1, env=countauthor,             keyword=biblioauthor]%
        \printbibliography[heading=nobibheading, section=1, env=countauthorvakscopuswos, filter=vakscopuswos]%
        \printbibliography[heading=nobibheading, section=1, env=countauthorscopuswos,    filter=scopuswos]}%
        %
        \nocite{*}%
        %
        {\publications} Основные результаты по теме диссертации изложены в~\arabic{citeauthor}~печатных изданиях,
        \arabic{citeauthorvak} из которых изданы в журналах, рекомендованных ВАК%
        \ifnum \value{citeauthorscopuswos}>0%
            , \arabic{citeauthorscopuswos} "--- в~периодических научных журналах, индексируемых Web of~Science и Scopus%
        \fi%
        \ifnum \value{citeauthorconf}>0%
            , \arabic{citeauthorconf} "--- в~тезисах докладов.
        \else%
            .
        \fi%
        \ifnum \value{citeregistered}=1%
            \ifnum \value{citeauthorpatent}=1%
                Зарегистрирован \arabic{citeauthorpatent} патент.
            \fi%
            \ifnum \value{citeauthorprogram}=1%
                Зарегистрирована \arabic{citeauthorprogram} программа для ЭВМ.
            \fi%
        \fi%
        \ifnum \value{citeregistered}>1%
            Зарегистрированы\ %
            \ifnum \value{citeauthorpatent}>0%
            \formbytotal{citeauthorpatent}{патент}{}{а}{}%
            \ifnum \value{citeauthorprogram}=0 . \else \ и~\fi%
            \fi%
            \ifnum \value{citeauthorprogram}>0%
            \formbytotal{citeauthorprogram}{программ}{а}{ы}{} для ЭВМ.
            \fi%
        \fi%
        % К публикациям, в которых излагаются основные научные результаты диссертации на соискание учёной
        % степени, в рецензируемых изданиях приравниваются патенты на изобретения, патенты (свидетельства) на
        % полезную модель, патенты на промышленный образец, патенты на селекционные достижения, свидетельства
        % на программу для электронных вычислительных машин, базу данных, топологию интегральных микросхем,
        % зарегистрированные в установленном порядке.(в ред. Постановления Правительства РФ от 21.04.2016 N 335)
    \end{refsection}%
    \begin{refsection}[bl-author, bl-registered]
        % Это refsection=2.
        % Процитированные здесь работы:
        %  * попадают в авторскую библиографию, при usefootcite==0 и стиле `\insertbiblioauthorimportant`.
        %  * ни на что не влияют в противном случае
        \nocite{vakbib2}%vak
        \nocite{patbib1}%patent
        \nocite{progbib1}%program
        \nocite{bib1}%other
        \nocite{confbib1}%conf
    \end{refsection}%
        %
        % Всё, что вне этих двух refsection, это refsection=0,
        %  * для диссертации - это нормальные ссылки, попадающие в обычную библиографию
        %  * для автореферата:
        %     * при usefootcite==0, ссылка корректно сработает только для источника из `external.bib`. Для своих работ --- напечатает "[0]" (и даже Warning не вылезет).
        %     * при usefootcite==1, ссылка сработает нормально. В авторской библиографии будут только процитированные в refsection=0 работы.
}

При использовании пакета \verb!biblatex! будут подсчитаны все работы, добавленные
в файл \verb!biblio/author.bib!. Для правильного подсчёта работ в~различных
системах цитирования требуется использовать поля:
\begin{itemize}
        \item \texttt{authorvak} если публикация индексирована ВАК,
        \item \texttt{authorscopus} если публикация индексирована Scopus,
        \item \texttt{authorwos} если публикация индексирована Web of Science,
        \item \texttt{authorconf} для докладов конференций,
        \item \texttt{authorpatent} для патентов,
        \item \texttt{authorprogram} для зарегистрированных программ для ЭВМ,
        \item \texttt{authorother} для других публикаций.
\end{itemize}
Для подсчёта используются счётчики:
\begin{itemize}
        \item \texttt{citeauthorvak} для работ, индексируемых ВАК,
        \item \texttt{citeauthorscopus} для работ, индексируемых Scopus,
        \item \texttt{citeauthorwos} для работ, индексируемых Web of Science,
        \item \texttt{citeauthorvakscopuswos} для работ, индексируемых одной из трёх баз,
        \item \texttt{citeauthorscopuswos} для работ, индексируемых Scopus или Web of~Science,
        \item \texttt{citeauthorconf} для докладов на конференциях,
        \item \texttt{citeauthorother} для остальных работ,
        \item \texttt{citeauthorpatent} для патентов,
        \item \texttt{citeauthorprogram} для зарегистрированных программ для ЭВМ,
        \item \texttt{citeauthor} для суммарного количества работ.
\end{itemize}
% Счётчик \texttt{citeexternal} используется для подсчёта процитированных публикаций;
% \texttt{citeregistered} "--- для подсчёта суммарного количества патентов и программ для ЭВМ.

Для добавления в список публикаций автора работ, которые не были процитированы в
автореферате, требуется их~перечислить с использованием команды \verb!\nocite! в
\verb!Synopsis/content.tex!.

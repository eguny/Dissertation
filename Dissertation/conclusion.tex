\chapter*{Заключение}                       % Заголовок
\addcontentsline{toc}{chapter}{Заключение}  % Добавляем его в оглавление

%% Согласно ГОСТ Р 7.0.11-2011:
%% 5.3.3 В заключении диссертации излагают итоги выполненного исследования, рекомендации, перспективы дальнейшей разработки темы.
%% 9.2.3 В заключении автореферата диссертации излагают итоги данного исследования, рекомендации и перспективы дальнейшей разработки темы.
%% Поэтому имеет смысл сделать эту часть общей и загрузить из одного файла в автореферат и в диссертацию:

В настоящей диссертационной работе изложены основные результаты исследования термодинамических, магнитных и фрустрационных свойств обобщенной модели Изинга на низкоразмерных решетках с учетом различных взаимодействий между ближайшими и вторыми соседями, разных знаков обменных взаимодействий, декорирования и внешнего магнитного поля.

Определены точки и поля фрустраций, установлены критерии существования фрустраций в рассмотренных системах. Впервые в мировой литературе выведено точное аналитическое решение модели Изинга на  декорированной решетке при наличии магнитного поля. Также, выведены точные выражения для намагниченностей и энтропий при нулевой температуре.

Цель работы состояла в рассмотрении обобщенной модели Изинга с произвольным количеством различных обменных взаимодействий как между ближайшими, так и между вторыми соседями с учетом декорирования и магнитного поля на одномерной цепочке, а также обобщенной модели Изинга с четырьмя различными обменными взаимодействиями между ближайшими соседями на квадратной решетке в том числе с учетом декорирования, но в отсутствие магнитного поля, с последующим изучением их термодинамических, магнитных и фрустрационных свойств.

Для выполнения поставленной цели были решены следующие задачи:
\begin{enumerate}
  	\item Выведена трансфер-матрица Крамерса--Ваннье, обобщенная на произвольное число трансляций решетки, являющаяся произведением трансфер-матриц с одной трансляцией;
	\item Исследованы термодинамические, магнитные и фрустрационные свойства обобщенной модели Изинга на одномерной цепочке, в том числе при учете декорирования;
	\item Выведено точное аналитическое выражение для свободной энергии Гельмгольца обобщенной модели Изинга на квадратной решетке с двумя трансляциями с применением комбинаторного метода Вдовиченко--Фейнмана.
	\item Исследованы термодинамические и фрустрационные свойства обобщенной модели Изинга на квадратной решетке, в том числе при учете декорирования, но в отсутствие магнитного поля.
\end{enumerate}

Несмотря на то, что была проделана значительная работа, еще многое остается неясным: в частности, не ясна причина возникновения и сохранения дополнительного малого пика теплоемкости при рассмотрении фрустрированных состояний обобщенной модели Изинга на квадратной решетке.

Ценность данной работы заключается в полученных точных решениях усовершенствованной, так называемой обобщенной, модели Изинга, анализ которых позволил установить совершенно новые фрустрационные особенности спиновых систем.

Дальнейшее развитие представленной работы направлено на получение точных решений на других обобщенных решетках (треугольная, гексагональная, кагоме и другие), а также в анализе термодинамических, магнитных и фрустрационных свойств этих моделей.

В заключение автор выражает благодарность и большую признательность научному руководителю доктору физико-математических наук Кассан-Оглы Феликсу Александровичу за предоставленную тему настоящей диссертации и конструктивную помощь в ее решении. 
